\documentclass{article}
\usepackage[T2A]{fontenc}
\usepackage[utf8]{inputenc}   
\usepackage[english, russian]{babel}

% Set page size and margins
% Replace `letterpaper' with`a4paper' for UK/EU standard size
\usepackage[a4paper,top=2cm,bottom=2cm,left=2cm,right=2cm,marginparwidth=1.75cm]{geometry}

\usepackage{amsmath}
\usepackage{graphicx}
\usepackage[colorlinks=true, allcolors=blue]{hyperref}
\usepackage{amsfonts}
\usepackage{amssymb}
% \usepackage[left=1cm,right=1cm,top=1cm,bottom=1cm]{geometry}
\usepackage{hyperref}
\usepackage{seqsplit}
\usepackage[dvipsnames]{xcolor}
\usepackage{enumitem}
\usepackage{algorithm}
\usepackage{algpseudocode}
\usepackage{algorithmicx}
\usepackage{mathalfa}
\usepackage{mathrsfs}
\usepackage{dsfont}
\usepackage{caption,subcaption}
\usepackage{wrapfig}
\usepackage[stable]{footmisc}
\usepackage{indentfirst}
\usepackage{rotating}
\usepackage{pdflscape}

\usepackage{MnSymbol,wasysym}
\usepackage{minted}

\begin{document}

\begin{center}
\Large {Задание 1. Асимптотические сложности.}
\end{center}

\bigskip

\textbf{1} Известно, что $f(n) = O(n^2),\ g(n) = \Omega(1),\ g(n) = O(n)$. Положим $$h(n) = \cfrac{f(n)}{g(n)}.$$ 

1. Возможно ли, что \textbf{а)} $h(n) = \Theta(n\log n)$; \textbf{б)} $h(n) = \Theta(n^3)$?

2. Приведите наилучшие (из возможных) верхние и нижние оценки на функцию $h(n)$ и приведите пример функций $f(n)$ и $g(n)$ для которых ваши оценки на $h(n)$ достигаются.

\medskip

\textbf{2} Найдите $\Theta$-асимптотику $\sum\limits_{i=1}^n \sqrt{i^3+2i+5}$.

\medskip

\textbf{3} Докажите, что асимптотика $\sum\limits^n_{i=1} i^{\alpha}
= \Theta(n^{1+\alpha})$, если
$\alpha > 0$.

\medskip

\textbf{4} Найдите $\Theta$-асимптотику функции
$g(n) = \sum\limits^n_{k=1} \frac{1}{k};$

\medskip

\textbf{4} Найдите $\Theta$-асимптотику функции $f(n) = \sum\limits^n_{k=0} {n \choose k}$

\medskip

\textbf{5} Оцените асимптотически, сколько раз будет напечатана строка ''heh'' при вызове функции $\text{f}$.

\begin{minted}{python}
    def f(n):
        for i = 1; i < n; i *= 2:
            for j in range(i*i*i):
                print("heh")

        for i = 1; i < n; i += 2:
            for j in range(i*i):
                print("heh")
\end{minted}

\textbf{6} Оцените асимптотику роста функции $f(n) = 1 + c + c^2 + \dots + c^n$ в зависимости от параметра $c$.

\medskip

\end{document}