\documentclass{article}
\usepackage[T2A]{fontenc}
\usepackage[utf8]{inputenc}   
\usepackage[english, russian]{babel}

% Set page size and margins
% Replace `letterpaper' with`a4paper' for UK/EU standard size
\usepackage[a4paper,top=2cm,bottom=2cm,left=2cm,right=2cm,marginparwidth=1.75cm]{geometry}

\usepackage{amsmath}
\usepackage{graphicx}
\usepackage[colorlinks=true, allcolors=blue]{hyperref}
\usepackage{amsfonts}
\usepackage{amssymb}
% \usepackage[left=1cm,right=1cm,top=1cm,bottom=1cm]{geometry}
\usepackage{hyperref}
\usepackage{seqsplit}
\usepackage[dvipsnames]{xcolor}
\usepackage{enumitem}
\usepackage{algorithm}
\usepackage{algpseudocode}
\usepackage{algorithmicx}
\usepackage{mathalfa}
\usepackage{mathrsfs}
\usepackage{dsfont}
\usepackage{caption,subcaption}
\usepackage{wrapfig}
\usepackage[stable]{footmisc}
\usepackage{indentfirst}
\usepackage{rotating}
\usepackage{pdflscape}

\usepackage{MnSymbol,wasysym}
\usepackage{minted}

\begin{document}

\begin{center}
\Large {Задание 3. Атомарные битовые операции. k-я порядковая статистика. Алгоритм Евклида.}
\end{center}

\bigskip

\textbf{Указание 1:} в этом задании (кроме задачи 6) мы будем использовать алгоритм поиска $k$-й порядковой статистики в качестве черного ящика, принимающего массив и число $k$, и за линейное время находящего элемент, который будет стоять на $k$-м месте в отсортированном массиве. Полный вывод и реализация алгоритма будут в видео, которое будет выложено в группу, к задаче 6 можно приступать после этого.

\medskip

\textbf{Указание 2:} для выполнения второй половины этого задания (задачи 6-10) понадобится материал следующего занятия, поэтому если сейчас что-то совсем непонятно, это нормально.

\medskip

\textbf{1} На прямой задано $n$ строго вложенных отрезков в виде пар концов $(l_i, r_i)$. Они могут поступать на вход в произвольном порядке. Постройте алгоритм, находящий \textbf{множество точек на прямой}, покрытое ровно $\lceil \frac{2n}{3} \rceil$ отрезками.

\medskip

\textbf{2} На доске написан набор положительных целых чисел. За один ход можно взять любые два числа и вычесть из большего меньшее. Процесс останавливается, когда все числа становятся одинаковыми. Докажите, что этот процесс всегда остановится. Какие числа останутся в результате?

\medskip

\textbf{3} Оцените сложность алгоритма Divide, приведенного на странице 19 книги Дасгупты.

\medskip

\textbf{4} Предложите алгоритм возведения $n$-битовых чисел в степень по модулю, оцените его сложность.

\medskip

\textbf{5} Предложите эффективный алгоритм вычисления наименьшего общего кратного (НОК) двух чисел в битовой модели вычислений (время выполнения операций зависит от длины битовой записи чисел).

\medskip

\textbf{6} С какой асимптотикой будет работать алгоритм поиска $k$-й порядковой статистики, если делить массив на группы не по пять элементов, а по три? По семь?

\medskip

\textbf{7} Найдите представление НОД чисел $a = 36$ и $b = 45$ в виде их линейной комбинации, то есть таких чисел $x$ и $y$, что $ax + by = gcd(a, b)$. Воспользуйтесь расширенным алгоритмом Евклида для решения этой задачи.

\medskip

\textbf{8} Решите уравнения в целых числах. Нужно найти все решения, а не только частное.

\begin{enumerate}
    \item $238x + 385y = 133$
    \item $143x + 121y = 52$
\end{enumerate}

\medskip

\textbf{9} Решите сравнение $68x + 85 \equiv 0 (\mod 561)$ с помощью расширенного алгоритма Евклида. Требуется найти все решения в вычетах.

\medskip

\textbf{10} Найдите обратный остаток $7^{-1} (\mod 102)$

\end{document}